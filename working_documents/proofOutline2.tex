\documentclass[11pt]{article}
% Include statements
\usepackage{graphicx}
\usepackage{amsfonts,amssymb,amsmath,amsthm}
\usepackage[numbers,square]{natbib}
\usepackage[left=1in,top=1in,right=1in,bottom=1in,nohead]{geometry}
\usepackage[all]{xy}
\usepackage{multirow,rotating,array}
\usepackage[ruled,lined]{algorithm2e}
\SetKw{KwSet}{Set}
%\usepackage{algorithm,algorithmic}
\usepackage{pdfsync}
\usepackage{setspace}
\usepackage{bbm}
\usepackage{moresize}
\usepackage{hyperref}
\hypersetup{backref,colorlinks=true,citecolor=blue,linkcolor=blue,urlcolor=blue}
\renewcommand{\qedsymbol}{$\blacksquare$}

% Bibliography
\bibliographystyle{plainnat}

% Theorem environments
\usepackage{aliascnt}

\newtheorem{theorem}{Theorem}[section]

\newaliascnt{result}{theorem}
\newtheorem{result}[theorem]{Result}
\aliascntresetthe{result}
\providecommand*{\resultautorefname}{Result}
\newaliascnt{lemma}{theorem}
\newtheorem{lemma}[lemma]{Lemma}
\aliascntresetthe{lemma}
\providecommand*{\lemmaautorefname}{Lemma}
\newaliascnt{prop}{theorem}
\newtheorem{proposition}[prop]{Proposition}
\aliascntresetthe{prop}
\providecommand*{\propautorefname}{Proposition}
\newaliascnt{cor}{theorem}
\newtheorem{corollary}[cor]{Corollary}
\aliascntresetthe{cor}
\providecommand*{\corautorefname}{Corollary}
\newaliascnt{conj}{theorem}
\newtheorem{conjecture}[conj]{Conjecture}
\aliascntresetthe{conj}
\providecommand*{\conjautorefname}{Corollary}
\newaliascnt{def}{theorem}
\newtheorem{definition}[def]{Definition}
\aliascntresetthe{def}
\providecommand*{\defautorefname}{Definition}

\newtheorem{assumption}{Assumption}
\renewcommand{\theassumption}{\Alph{assumption}}
\providecommand*{\assumptionautorefname}{Assumption}

\def\algorithmautorefname{Algorithm}
\renewcommand*{\figureautorefname}{Figure}%
\renewcommand*{\tableautorefname}{Table}%
\renewcommand*{\partautorefname}{Part}%
\renewcommand*{\chapterautorefname}{Chapter}%
\renewcommand*{\sectionautorefname}{Section}%
\renewcommand*{\subsectionautorefname}{Section}%
\renewcommand*{\subsubsectionautorefname}{Section}% 


% Macros
\def\indep{\perp\!\!\!\perp}
\newcommand{\given}{\ \vert\ }
\newcommand{\F}{\mathcal{F}}
\newcommand{\E}{\mathbb{E}}
\newcommand{\Expect}[1]{\E\left[ #1 \right]}
\renewcommand{\P}{\mathbb{P}}
\newcommand{\R}{\mathbb{R}}
\newcommand{\norm}[1]{\lVert #1 \rVert}
\newcommand{\email}[1]{\href{mailto:#1}{#1}}
\newcommand{\X}{\mathbb{X}}
\DeclareMathOperator*{\argmin}{argmin}
\DeclareMathOperator*{\trace}{tr}

\newcommand{\A}{\mathcal{A}}
\renewcommand{\S}{\mathcal{S}}
\newcommand{\T}{\mathcal{T}}
\newcommand{\D}{\mathcal{D}}
\newcommand{\x}{\mathbf{x}}
\newcommand{\PP}{\mathcal{P}}


\begin{document}
\noindent\textbf{\sc DW
        \hfill Possible proof technique
        \hfill \today}
\rule{6.5in}{1pt}
\tableofcontents
\newpage

\section{Table for matrix sketching results}
\begin{table}[!h]
\begin{tabular}{l | l | l | l | l | l}
& Covariance estimation & Eigenvalues & Eigenvectors& PCR ($\hat{Y}$) & PCR ($\hat{\beta}$) \\
& ($\E\norm{\hat{\Sigma} - \Sigma}_2^2$ or canonical angle) & & &\\
\hline
Nystrom &&&&&\\
CS && $CS_2$ & $CS_3$ && $CS_5$\\
Martinsson&&&&&
\end{tabular}
\caption{Note that the CS results apply to AIMER.}
\end{table}



\section{Preliminary notation, definitions, and statistical model}
\subsection{Notation}
\begin{itemize}
\item $\X \in \R^{n\times p}$
\item $\x_j = [X_{i1}, \ldots, X_{ip}]^{\top} \in \R^{p}$ is the $j^{th}$ column of $\X$ and an i.i.d. sample from 
$x_j \sim N(0,\Sigma(j,j))$.
\item $\PP = \{1,\ldots,p\}$ 
\item $\A = \{$ of active covariates $\}$
\item $\S = \{$ nonzero marginal covariance $\}$ (using $\S$ as it is the `selected' model)
\item $\D = \A \setminus \S$ (to be the difference between active and selected covariates)
\item $\T = \{$ nonzero $\theta$ $\}$ (using $\T$ due to...  whatever)
\item For any subsets $A,B$ of $\PP$ and matrix $\mathbb{A}$, the submatrix with rows $A$ and columns $B$ is $\mathbb{A}_{A,B}$

\end{itemize}
\subsection{Definitions}
\begin{itemize}
\item
The underlying machinery of these supervised PCA papers is a suite of estimators of the form $\hat\Sigma_{A,B}$, where $A,B \subseteq \PP$.  In the SPCA
paper, they choose $\hat\Sigma_{\S,\S}$.  Using $F$ is tantamount to using $\hat\Sigma_{\PP,\S}$.  This protects somewhat against $\S \subset \A$.  If we had a good estimator of $\T$
we would/could use $\hat\Sigma_{\T,\S}$ instead.  Perhaps this estimator should be investigated as well...

\item $F = \X^{\top}\X_1 = V(F) \Lambda(F) U(F)^{\top}$ (note, I think this reversed order makes much more sense at we are looking at approximating $\X^{\top}\X = VD^2V^{\top}$...)
I haven't included any normalization by a function of $n$, which is surely necessary to get convergence.  In particular, the sample covariance would be $n^{-1}\X^{\top}\X$,
so defining $F \leftarrow n^{-1}F$ would seemingly make sense.
\end{itemize}




\subsection{Model}
\begin{itemize}
\item 
Let the covariance matrix for $X$ be
\begin{equation}
\Sigma 
= 
 \begin{bmatrix} 
 \Sigma_{1} & 0  \\ 0 & \Sigma_2 
 \end{bmatrix},
\end{equation}
where $\Sigma_{1} \equiv \Sigma_{\A,\A} = \Theta \Lambda \Theta^{\top} + \sigma^2I = \sum_{m=1}^M \lambda_m \theta_m \theta_m^\top + \sigma^2 I$.  
This should be equivalent to the model in the next bullet if $\Sigma_2 = \sigma^2 I$.  We can probably generalize
this model to let $\Sigma_2$ have its own eigenvector structure (with eigenvalues strictly smaller than $\Lambda$.
\item
$
X_{ij} = \sum_{m=1}^M \lambda_m^{1/2}\eta_{im} \theta_{jm} + \sigma z_{ij}
$
where $\norm{\theta_{m}}_2^2 = 1$ and $\theta_{ m} \equiv \theta_{\cdot m}$, $\langle \theta_m, \theta_{m'} \rangle = 0$ if $m \neq m'$, and $\theta_{jm} = 0$ if $j \notin \A$.  All $z_{ij}$ and $\eta_{im}$ are standard normals and mutually independent.
\item The regression model: 
\begin{equation}
Y_i = \beta_0 + \sum_{m=1}^{\tilde{M}} \beta_m  \eta_{im} + W_i.
\label{eq:YregModel}
\end{equation}
Here, I write $\tilde{M}$ to indicate the this may be different than $M$.  
\end{itemize}

\subsection{Assumptions}
\label{sec:assumptions}

\begin{itemize}
\item $(\sum_{m = 1}^M \lambda_m \theta_{jm} \theta_{km})^2 \leq \gamma_n$ for $k \in \D$.
\item We can estimate $\sigma^2$ well so we consider it known (really, just to simplify things so we can just subtract off the diagonal component before hand)
\item $\lambda_{\max} \leq C_{\Lambda}$ independent of $n$
\item Probably eventually will need to codify the rates for size of some of the above sets (e.g. $|\A| \asymp a_n$)
\end{itemize}

\newpage

\section{Showing $CS_2$}
\subsection{Overview}
Suppose $A \in \R^{m\times n}$ and $\tilde{A} = A + E$.  Also, $U^{\top}AV = \begin{bmatrix} D \\ 0 \end{bmatrix}$ and $\tilde{U}^{\top}\tilde{A}\tilde{V} = \begin{bmatrix} \tilde{D} \\ 0 \end{bmatrix}$.
There is a main theorem for bounding the distance between $D$ and $\tilde{D}$:

\begin{theorem}[Mirsky]
\[
\norm{\tilde{D} - D} \leq \norm{E}
\]
where the norm can be any unitarily equivalent norm (e.g. $\norm{\cdot}_2$ or $\norm{\cdot}_F$).
\label{thm:mirsky}
\end{theorem}

Ultimately, we will probably use the following $\forall k$:
\[
|D_k - \tilde{D}_k | \leq \norm{E}_F
\]



\subsection{The result}
To start, write $M_{ij} :=  \sum_{m=1}^M \lambda_m^{1/2}\eta_{im} \theta_{jm}$.  Then $M_{ij} = 0$ if $j \notin \A$.
\begin{align}
F & = \left[\sum_{i=1}^n X_{ij} X_{ik}\right]_{j \in \PP, k \in \S}\\
& = \left[\sum_{i=1}^n(\sum_{m=1}^M \lambda_m^{1/2}\eta_{im} \theta_{jm} + \sigma z_{ij})
(\sum_{m=1}^M \lambda_m^{1/2}\eta_{im} \theta_{jm} + \sigma z_{ij})\right]_{j \in \PP, k \in \S} \\
& = 
\begin{bmatrix}
\sum_{i=1}^n(M_{ij} + \sigma z_{ij})(M_{ik} + \sigma z_{ik})\\
\sum_{i=1}^n(\sigma z_{ij})(M_{ik} + \sigma z_{ik})
\end{bmatrix} \\
& 
=
\begin{bmatrix}
\sum_{i=1}^n(M_{ij}M_{ik}+ \sigma z_{ij}M_{ik} + \sigma z_{ik}M_{ij} +  \sigma^2 z_{ij}z_{ik})\\
\sum_{i=1}^n( \sigma z_{ij}M_{ik} +  \sigma^2 z_{ij}z_{ik})
\end{bmatrix},
\end{align}
where the top block has $j \in \A$ and the bottom block has $j \in \A^c$ (this convention will persist for the rest of this proof).

Using the result from Theorem \ref{thm:mirsky}, write 
$A = \tilde{F}$  and $E = \Sigma_{\PP,\D} - \tilde{F}$, where $\tilde{F} = [F |  0] \in \R^{p \times |\A|}$.
The nonzero singular values of $F$ and $\tilde{F}$ are identical. 
 Hence, the approximation error in the estimation of the singular values of $\Sigma_1$  will be encoded
 in the difference $\Sigma_{\PP,\D} - \tilde{F}$.

Writing 
 $E = \tilde{\Sigma}_{\PP,\S} - \tilde{\Sigma}_{\PP,\S} + \Sigma_{\PP,\D} - \tilde{F}$, 
where 
\[
\tilde{\Sigma}_{\PP,\S} 
= 
\begin{bmatrix} 
\Sigma_{\A,\S} & 0 \\
0 & 0
\end{bmatrix}
\in
\R^{p \times |\A|},
\]  
then
\[
\norm{E}_F \leq \norm{\tilde{\Sigma}_{\PP,\S} - \tilde{F}}_F  + \norm{\Sigma_{\PP,\D} - \tilde{\Sigma}_{\PP,\S}}_F.
\]
We should have $\E \tilde{F} = n \tilde{\Sigma}_{\PP,\S}$ and hence should be able to control the first term with concentration or
convergence results.  The second term will have an irreducible error given by
\[
\norm{\Sigma_{\PP,\D} - \tilde{\Sigma}_{\PP,\S}}_F^2
= 
\sum_{j \in \A, k \in \D} \Sigma_{j,k}^2
= 
 \sum_{j \in \A, k \in \D}\left( \sum_{m=1}^M \lambda_m \theta_{jm}\theta_{km}\right)^2
 \leq
 |\A| |\D| \gamma_n
\]
under the assumptions in Section \ref{sec:assumptions}. 

\subsection{Old material assuming one latent factor (needs updating)}


\textbf{Start: delete this later.  I'm including it to facilitate later translation.}

The goal here is to extend the results of the SPCA paper by including the possibility in $\Sigma_1$ that we have missed some important features  (hence their $\Sigma_1$ corresponds 
to our $\Sigma_{11}$.  The model for $X$ is then (here I'm writing/thinking about a single latent factor model, I'm presuming that complexifying that to multi-factor will be a matter
of notation):
\[
X_{ij} = v_i \theta_j + \sigma z_{ij}
\]
where $v_i,z_{ij}$ are all mutually independent standard normals and $\theta_j \neq 0$ iff $j \in \S$.

\noindent\textbf{End: delete}

Note the expectation of $F$:
\[
\E F =
\begin{bmatrix}
n \theta\theta^{\top} + n\sigma^2 I \\
0
\end{bmatrix}.
\]
Hence, 
\[
E = 
\begin{bmatrix}
F - \E F
\end{bmatrix}.
-
\begin{bmatrix}
n \theta\theta^{\top} + n\sigma^2 I \\
0
\end{bmatrix}.
\]
So, up to the $n\sigma^2I$ factor, we have to bound the norm difference between a random matrix and it's expectation.
\begin{align}
F - \E F -
 \begin{bmatrix}
n \theta\theta^{\top} + n\sigma^2 I \\
0
\end{bmatrix}
 & =
\begin{bmatrix}
\sum_{i=1}^nv_i^2\theta_j\theta_k  - n \theta_j\theta_k\\
0
\end{bmatrix}
+
\begin{bmatrix}
\sigma \sum_{i=1}^n v_i(z_{ij}\theta_k + z_{ik}\theta_j) \\
\sigma \sum_{i=1}^n v_iz_{ij}\theta_k + \sigma^2 \sum_{i=1}^n z_{ij}z_{ik}
\end{bmatrix}
+ \\
& \qquad + 
\begin{bmatrix}
 \sigma^2 \sum_{i=1}^n z_{ij}z_{ik} - n\sigma^2 I \\
0
\end{bmatrix} \\
& = (i) + (ii) + (iii).
\end{align}
Now,
\[
(i) = 
(\sum_{i=1}^nv_i^2 -n)
\begin{bmatrix}
\theta\theta^{\top}\\
0
\end{bmatrix}
\]
and
\[
(iii) = 
\sigma^2 
\begin{bmatrix}
\sum_{i=1}^n Z_{i\S}Z_{i\S}^{\top} - n I \\
0
\end{bmatrix},
\]
where $Z_{i\S} = [z_{ij}]_{j \in \S}$. Similarly, $Z_{i\S^c} = [z_{ij}]_{j \in \S^c}$.  Hence, 
\[
(ii) = 
\sigma \sum_{i=1}^n v_i
\begin{bmatrix}
 Z_{i\S}\theta^{\top}\\
 Z_{i\S^c}\theta^{\top}
\end{bmatrix}
+
\sigma \sum_{i=1}^n v_i
\begin{bmatrix}
\theta Z_{i\S}^{\top}\\
0
\end{bmatrix}
+
\sigma^2 \sum_{i=1}^n 
\begin{bmatrix}
0 \\
Z_{i\S} Z_{i\S^c}^{\top}\\
\end{bmatrix}.
\]
Now, concentration results can be used to show $E$ is small in Frobenius norm.  Presumably, we can find some results about spectral norm as well (which would
probably be more useful as it would allow us to say $|d_j - \tilde{d_j}| \leq \norm{E}_2$ for all $j$).

\section{Showing $CS_3$}
Using the result that
\[
\norm{\hat{v} - v}_2^2 \leq 2 \sin(\angle (\hat{v},v) )
\]
we can do the following.  Supposing that 
$\Sigma = [\tilde{\Sigma}_\S | \tilde{\Sigma}_{\S^c}]$, $F = V(F) D(F) U(F)^{\top}$, $\tilde{\Sigma} = V DU^{\top}$, and $\Sigma = \Theta \Lambda \Theta^{\top}$, then for $k \in \S$,
\[
\norm{v_q(F) - \theta_q}_2 \leq  \norm{v_q(F) - v_q}_2 + \norm{v_q - \theta_q}_2 \leq \sqrt{2}\left( \sin(\angle (v_q(F), v_q)) + \sin(\angle (v_q,\theta_q))\right).
\]
So, by Yu et al. (2015)\footnote{\url{http://www.statslab.cam.ac.uk/~yy366/index_files/Biometrika-2015-Yu-biomet_asv008.pdf}}, Theorem 3
\[
 \sin(\angle (v_q(F), v_q)) 
 \leq 
 2\frac{(2d_{\max} + \norm{F - \tilde{\Sigma}}_{op})\min\{ \norm{F - \tilde{\Sigma}}_{op}, \norm{F - \tilde{\Sigma}}_{F}\}}{\tilde{\delta}_q},
\]
where $\tilde{\delta}_q = \min\{d_{q-1} - d_q, d_{q} - d_{q+1}\}$, which will be controlled by assumption on $\Sigma$ (for instance, $d_{\max} \leq \lambda_{\max}$).

Now, looking at $\norm{F - \tilde{\Sigma}}_{F}^2$ component wise for $j \in p$ and $k \in \S$
\[
(F(j,k) - \tilde{\Sigma}(j,k))^2 = (\x_j^{\top}\x_k - \E x_jx_k)^2.
\]
This will be controllable via asymptotics or concentration. 

There will be nonzero approximation bias if $\D \neq \emptyset$.  Using the same result as above
\[
\sin(\angle (v_q,\theta_q))
\leq 
2\frac{(2\lambda_{\max} + \norm{ \tilde{\Sigma} - \Sigma}_{op})\min\{\norm{\tilde{\Sigma} - \Sigma}_{op}, \norm{\tilde{\Sigma} - \Sigma}_{F}\}}{\delta_q},
\]
where $\delta_q = \min\{\lambda_{q-1} - \lambda_q, \lambda_{q} - \lambda_{q+1}\}$.  This quantity will again be controlled by assumption on $\Sigma$. 

Now, looking at $\norm{\tilde{\Sigma} - \Sigma}_{F}^2$ component wise for $j,k \in \PP$
\[
( \tilde{\Sigma}(j,k)) - \Sigma(j,k))^2 
= 
\begin{cases}
0 & \textrm{ if } k \in \S \\
(\sum_{m = 1}^M \lambda_m \theta_{jm} \theta_{km})^2  & \textrm{ if } j \in \A,k \in \D \\
(\sum_{m = 1}^M \lambda_m \theta_{jm} \theta_{km}  + \sigma^2)^2 & \textrm{ if } j = k \in \D \\
(\sigma^2)^2 & \textrm{ if } j = k \notin \A \\
\end{cases}
\]
Now, we might make some assumptions about the size of this ``residual'' components, due to a norm constraint on these components implying a norm constraint on the $\beta$'s. 

Some such assumptions are listed in Section \ref{sec:assumptions}. Then
\[
\norm{\tilde{\Sigma} - \Sigma}_{F}^2 \leq |\A||\D|\gamma_n,
\]
which implies that
\[
\sin(\angle (v_q,\theta_q))
\leq 
2\frac{(2\lambda_{\max} + |\A||\D|\gamma_n)|\A||\D|\gamma_n}{\delta_q},
\]

\section{Showing $CS_5$}
%\section{Proof outline/sketch}
\begin{enumerate}
\item Show that $v_m(F)$ is close to $\theta_m$ (the PC loadings) and $\lambda_m(F)$ is close to $\lambda_m$
\begin{enumerate}
\item This is the topic of the document ``convergenceSingularVectorsValues.pdf''.  We need show that $v_m(F)$ converges to $\theta_m$.  So, perhaps, $v_m(F) = \theta_m + \delta_m$,
where $\norm{\delta_m}$ is small (note: we need to formalize the connection between bounded $\sin($ canonical angles$)$ of singular vectors and writing them in the fashion.  Perhaps
the asymptotic expansion is more amenable?)

\end{enumerate}

\item The regression part of the procedure regresses $Y$ onto the PC scores, which are the coordinates in the PC, given by $\hat{u}_m = \X v_m(F) \lambda_m^{-1/2}(F)$.  We need to show that these coordinates aren't too far from the coordinates created by inner product with $\theta_{m'}$:
\begin{equation}
\left\langle \sum_{m=1}^M \eta_{im} \theta_{m}, \theta_{m'} \right\rangle = \eta_{i,m'} \lambda_{m'}
\end{equation}
\begin{enumerate}
\item This can be done via inserting the model for $X$ in for $\X$ in the definition of $\hat{u}_k$.
\begin{equation}
\X v_k(F) 
=
\begin{bmatrix} 
\sum_{j=1}^p \left(\sum_{m=1}^M \lambda_m^{1/2}\eta_{1m} \theta_{jm} + \sigma z_{1j}\right) v_{jk}(F) \\
\vdots \\
\sum_{j=1}^p \left(\sum_{m=1}^M \lambda_m^{1/2}\eta_{nm} \theta_{jm} + \sigma z_{nj}\right)v_{jk}(F)
\end{bmatrix}
%=
%\begin{bmatrix} 
%\sum_{m=1}^M \eta_{1m} \theta_{m}^\top v_k(F) + \sigma z_{1}^{\top}v_{jk}(F) \\
%\vdots \\
%\sum_{m=1}^M \eta_{nm} \theta_{m}^\top v_k(F) + \sigma z_{n}^{\top}v_{jk}(F) 
%\end{bmatrix}
=
\sum_{m=1}^M
\lambda_m^{1/2}\theta_{m}^\top v_k(F) 
\begin{bmatrix} 
 \eta_{1m} \\
\vdots \\
 \eta_{nm} 
\end{bmatrix}
+
\sigma
\begin{bmatrix} 
z_{1}^{\top}v_{k}(F) \\
\vdots \\
 z_{n}^{\top}v_{k}(F) 
\end{bmatrix}.
\end{equation}
Using the approximation: $v_k(F) = \theta_k + \delta_k$,
\begin{equation}
\eta_{im} \theta_{m}^\top v_k(F)  = \eta_{im} \theta_{m}^\top (\theta_k + \delta_k) 
= 
\eta_{im} (\theta_{m}^\top \theta_k + \theta_{m}^\top\delta_k)
=
\begin{cases}
\eta_{ik}(1+\theta_{k}^\top\delta_k) & \textrm{ if } k = m \\
\eta_{im}(\theta_{m}^\top\delta_k) & \textrm{ if } k \neq m
\end{cases}
\end{equation}

\begin{enumerate}
\item Fix $k \neq m$: 
\begin{equation}
\eta_{im}\lambda_m^{1/2}\theta_{m}^\top v_k(F) \lambda_k^{-1/2}(F) = \left(\frac{\lambda_m}{\lambda_k(F)}\right)\eta_{im}(\theta_{m}^\top\delta_k)
\end{equation}
So, we need the ratio of eigenvalues to be bounded and then perhaps
\begin{equation}
|\theta_{m}^\top\delta_k| \leq \norm{\delta_k}_2 = o(\textrm{some rate}).
\label{eq:rhoDeltaInnerProduct}
\end{equation}
\item Fix $k = m$: 
\begin{equation}
\eta_{ik}\lambda_k^{1/2}\theta_{k}^\top v_k(F) \lambda_k^{-1/2}(F) = \left(\frac{\lambda_k}{\lambda_k(F)}\right)\eta_{ik}(1+\theta_{k}^\top\delta_k)
\end{equation}
Now, we need the ratio of eigenvalues to go to one (implied by the perturbation bound?) and using the above bound in equation \eqref{eq:rhoDeltaInnerProduct}:
\begin{equation}
\left(\frac{\lambda_k}{\lambda_k(F)}\right)\eta_{ik}(1+\theta_{k}^\top\delta_k) \rightarrow \eta_{ik}
\end{equation}

\end{enumerate}
\item Combining (i) and (ii) 
\begin{equation}
\sum_{m=1}^M
\lambda_m^{1/2}\theta_{m}^\top v_k(F) 
\begin{bmatrix} 
 \eta_{1m} \\
\vdots \\
 \eta_{nm} 
\end{bmatrix}
=
\begin{bmatrix} 
 \eta_{1k} \\
\vdots \\
 \eta_{nk} 
\end{bmatrix}
+ 
o(\textrm{some other rate})
\end{equation}
\item Lastly, we need to show that the measurement error term is bounded:
\[
\sigma
\begin{bmatrix} 
z_{1}^{\top}v_{k}(F) \\
\vdots \\
 z_{n}^{\top}v_{k}(F) 
\end{bmatrix}.
\]
This needs to be addressed with care as $z$ and $v$ are dependent.
\end{enumerate}
\item We need
to write down the form of the estimator: $\hat{U}_{\tilde{M}}^{\top}Y$.  
Plug in the regression model for $Y$ (equation \eqref{eq:YregModel}):
\begin{equation}
\hat\beta_m = \hat{u}_m^{\top}Y = \beta_0  \hat{u}_m^{\top}\mathbf{1} + \sum_{m=1}^{\tilde{M}} \beta_m  \hat{u}_m^{\top}\eta_m +  \hat{u}_m^{\top}W = \textrm{(a) + (b) + (c) }
\end{equation}

we need to write the regression model for $Y$ in terms of these estimated coordinates:
\begin{enumerate}
\item Maybe we can get rid of this via a max norm bound?
\begin{equation}
| \hat{u}_m^{\top}\mathbf{1}| \leq  \norm{\hat{u}_m}_1\norm{\mathbf{1}}_{\infty} =  \norm{\hat{u}_m}_1
\end{equation}
There should be something like a $n^{-1/2}$ running around.  So, this would require that $\norm{\hat{u}_m}_1 = o(n^{1/2})$, which isn't that likely.
\item Apply the above results that show that $\hat{u}_m \approx \eta_m$ and hence 
\[
 \beta_m  \hat{u}_m^{\top}\eta_m 
 \approx  \beta_m \norm{\eta_m}_2^2 
\]
So, if we have a $n^{-1}$ floating around, then $n^{-1}\norm{\eta_m}_2^2 \rightarrow 1$ and
\[
 \beta_m \norm{\eta_m}_2^2  \rightarrow \beta_m.
\]
\item $\hat{u}_m$ and $W$ are independent, so this can be shown to be small using a concentration bound (mean zero)
\end{enumerate}
\end{enumerate}
\bibliography{../SPCA.bib}
\end{document}

